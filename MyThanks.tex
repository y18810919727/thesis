% !Mode:: "TeX:UTF-8"

\chapter*{\centering  致 \space 谢}
\addcontentsline{toc}{chapter}{致谢}
在北京科技大学攻读博士的五年半对我来说意义深远。在这个过程中,我积累了知识、锻炼了心性、磨练了意志,知道了如何与人合作,明白了如何作为一个平凡的人参与伟大的事业。而我所获得的一切离不开老师、朋友、家人对我的帮助与支持。值此论文完成之际,对所有帮助过我的人表示感谢。

首先,最需要感谢的人是我的导师班晓娟教授。班老师除了在专业知识以及科研方法上提供了充足的指导外,
为我创造了得天独厚的研究环境以及高自由度的发展空间,让我能够倚着自己的兴趣和好奇心探索科学的边界。另外班老师的那句“读博士,不光是写几篇论文,而是全方位的锻炼与培养”更是我读博期间始终铭记于心的教诲,我也因此受益良多。

同样要感谢的是香港浸会大学的戴弘宁教授、挪威科技大学的王浩教授、吴狄教授、以及Miratlas SAS的王也弯老师。他们待我如同自己的亲学生,我在学术道路上的成长离不开各位老师不遗余力的帮助与指导。

还要感谢人工智能与三维可视化实验室的各位老师和同学。感谢姚超老师在研究方案设计、论文写作、毕业论文选题等方面提供的全方位指导。感谢张雅澜、马博渊、刘斯诺等师兄师姐给我树立了优秀的标杆指引我前进,并在我毕业、择业之际分享了宝贵的经验。同时感谢曹宇宁、何润姿、胡金龙、刘婷、李佳、周佳城、李潇睿、韩方圆、张子轩等一起参与过采矿充填项目的各位同学、师弟师妹们,跟你们一起去非洲出差、一起讨论项目实施方案、一起通宵达旦赶项目进度,是我永生难忘的日子。感谢实验室所有的老师同学,与你们交流、讨论让我学到无数宝贵的知识,因为你们让我在实验室的6年时间变得丰富多彩。

同时要感谢卢东旭、苏日娜、秦运慧、郑远硕四位博士,
% 与你们近一年来的陪伴与鼓励,与你们一起日子让我远离孤单,倍感温馨与快乐。
一年多来与你们在一起交流、吃饭、娱乐,让我远离孤单,真切地感受到了友情带给人的温馨与快乐。

最后还要感谢我的父母给我无条件的支持与关爱,让我没有任何后顾之忧地追求自己的学业。你们在我获得成就时陪伴我一起喝彩,在我最困难的时期作为我最坚实的后盾,我爱你们。

感谢非洲矿业有限公司沈家华、肖金林、姚松、王京伟等同仁在数据采集、算法测试期间提供的帮助。

感谢各位评审老师的认真评阅以及所提出的宝贵意见。

本论文承蒙国家重点研发计划项目(No.2019YFC0605300)资助,特此致谢。



