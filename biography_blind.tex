% !Mode:: "TeX:UTF-8"

\chapter*{\centering 作者简历及在学研究成果}

\noindent 一、作者入学前简历 \par
%博士入学前简历,根据自己情况填写,备注中填写职务等。
\vspace{2ex}
% \resizebox{0.8\linewidth}{!}{
\begin{tabular}{|c|c|c|}\hline
起止年月 & 学习或工作单位 & 备注 \\ \hline
2013年9月至2017年7月 & 
\begin{tabular}[c]{@{}c@{}}在北京科技大学计算机科学与技术 \\ 专业攻读学士学位\end{tabular}
 &  \\ \hline
% 2017年9月至2017年7月 & 在北京科技大学计算机科学与技术专业攻读学士学位 &  \\ \hline
% XXXX年XX月至XXXX年XX月 & 在XXXX 学校XXXX 专业攻读硕士学位 &  \\ \hline
% XXXX年XX月至XXXX年XX月 & 在XXXX 单位从事XXXX 岗位的工作 &  \\ \hline
\end{tabular} 
% }

\vspace{10pt}
\noindent 二、在学期间从事的科研工作 \par
%(应注明课题名称、参加身份、通过时间、通过方式、评定单位等)。
\begin{enumerate}
% \item 硕士博士毕业论文\LaTeX 模板的编写:\\
% 作为主要编辑人,从2016年3月份开始创建本项目以来一直进行修改,希望能制作一个满意的模板并且被官方认可。
\item 国家重点研发计划重点专项: 基于大数据的金属矿开采装备智能管控技术研发与示范(编号:2019YFC060XXXX), 主要参与人员,2019.12-2022.12。
\end{enumerate}
\noindent 三、在学期间所获的科研奖励 \par
%应注明奖励名称、授奖单位、授奖时间等,请填写科研方面奖励,请勿填写其他奖励信息,如不得填写三好研究生等奖励信息。)
\begin{enumerate}
    \item 2020年中国黄金协会科学技术一等奖
    \item 2022年北京科技大学太湖人才王道科技奖奖学金
    \item 2017年国际大学生程序设计竞赛ACM-ICPC亚洲区域赛(西安站)金奖
    \item 2017年国际大学生程序设计竞赛ACM-ICPC亚洲区域赛(青岛站)银奖
\end{enumerate}
\par

\noindent 四、在学期间发表的论文 \par
%应按照参考文献的格式来填写,包括编号。并在后面依次标明以下事项,各项之间用“.”分隔:1)标明“已发表”或“已录用”;2)是否“SCI/EI/STP/CSSCI 刊源”;3)是否被“SCI/EI/STP/CSSCI 检索”;4)检索号。第 2、3 项请标明具体检索名称)
\begin{enumerate}[label={[\arabic*]}]  %\setlength{\itemsep}{0pt}
% \begin{enumerate}[[1]]  %\setlength{\itemsep}{0pt}
% \begin{enumerate}
\item 第一作者. Autonomous-Jump-ODENet: Identifying Continuous-Time Jump Systems for Cooling-System Prediction[J]. IEEE Transactions on Industrial Informatics, 2022, doi: 10.1109/TII.2022.3207835.(已录用. SCI检索. 中科院分区小类1区. IF=11.648) 
\item 第一作者. Continuous-time prediction of industrial paste thickener system with differential ODE-net[J]. IEEE/CAA Journal of Automatica Sinica, 2022, 9(4): 686-698.(已检索. SCI检索.中科院分区小类1区. IF=7.847. 检索号:WOS:000766623800011)  
\item 第一作者. Integrated three-dimensional visualization and soft-sensing system for underground paste backfilling[J]. Tunnelling and Underground Space Technology, 2022, 127: 104578.(已检索. SCI检索. 中科院分区小类1区. IF=6.407. 检索号:WOS: 000811527500003) 
\item 第一作者. 基于强化学习的浓密机底流浓度在线控制算法[J]. 自动化学报, 2021, 47(7): 1558-1571.(已检索. EI刊源. CCF-A类中文期刊) 
\item 第一作者. A dual-attention recurrent neural network method for deep cone thickener underflow concentration prediction[J]. Sensors, 2020, 20(5): 1260. (已检索. SCI检索. 中科院分区小类3区. IF=3.847. 检索号:WOS: 000525271500022)  
\item 第一作者. Improving Word Representation Quality Trained by word2vec via a More Efficient Hierarchical Clustering Method[C]//International Conference on Cooperative Design, Visualization and Engineering. Springer, Cham, 2018: 299-303. (EI会议)
\item 第一作者. ODE-RSSM: Learning Stochastic Recurrent State Space Model from Irregularly Sampled Data. In Proceedings of the 37th AAAI Conference on Artificial Intelligence, 2023. (CCF-A类推荐英文会议,在审, phase 2)
% \item \textbf{Zhaolin Yuan}, et al. Visual-Based non-contact measurement for paste concentration[J]. Journal of Computing in Civil Engineering, xxxx, x(x): xx. (中科院SCI二区,在审)
\item 第二作者. An Improved Reinforcement Learning Based Heuristic Dynamic Programming Algorithm for Model-Free Optimal Control[C]//International Conference on Artificial Neural Networks. Springer, Cham, 2020: 282-294. (CCF-C类推荐会议) 
\item 第二作者. An industrial missing values processing method based on generating model[J]. Computer Networks, 2019, 158: 61-68.(已检索. SCI检索. 中科院分区小类3区. IF=5.493. 检索号:WOS:000472243200005)
\item 第三作者. 工业场景下基于深度学习的时序预测方法及应用[J]. 工程科学学报, 2022, 44(4): 757-766. doi: 10.13374/j.issn2095-9389.2021.12.02.004
\end{enumerate}
\par
\noindent 五、在学期间申请及授权的发明专利 \par

\begin{enumerate}[label={[\arabic*]}]  %\setlength{\itemsep}{0pt}
 \item 第二发明人. 一种基于强化学习的浓密机在线控制方法:中国,201910636652.7 [P]. 2019.11.01. 
 \item 第二发明人. 一种基于连续时间神经网络的浓密机预测控制方法及系统:中国, 202011493186.0 [P]. 2020.12.16. 
\item 第二发明人. 一种文本信息自动提取方法201810975598.4 [P]. 2018.08.24 
\item 第三发明人. 一种针对工业监测数据缺失的补全方法及补全装置:中国, 201910055378.4 [P]. 2020.09.25. 
 \item 第三发明人. 一种充填场景下的深锥浓密机智能控制方法:中国, 201910373119.6 [P]. 2019.08.06. 
 \item 第三发明人. 一种充填管道裂纹检测方法:中国, 202010544835.9 [P]. 2020.06.15.
\item 第四发明人. 一种膏体充填进度实时测量与可视化方法及系统:中国, 202010997613.2 [P]. 2020.09.21 
\item 第四发明人. 一种膏体浓度非接触式自动检测方法:中国, 202011026025.0 [P]. 2020.09.25. 
\end{enumerate}

%盲审论文,请隐去所有可能影响盲审结果的信息,诸如作者姓名、导师姓名、作者学号等。另外在此处,研究成果中论文作者的发表文章列表中应隐去所有作者的名字,只标明论文作者是第几作者,具体如“[第二作者].论文名称.……”