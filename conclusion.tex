\chapter{总结}
\section{本文研究工作总结}
% 本文针对ODE动态提出了
本文围绕基于微分方程网络的动态系统建模及预测技术开展研究,针对复杂动态系统存在的非线性、长时延、随机性、多阶段周期转换等特性,探索了采用微分方程网络建模系统动态并实现优化控制的基础理论及关键技术。

针对复杂动态系统存在的非线性、长时延特性,第三章开创性地提出了基于可微ODE-Net的高时延工业多输入输出系统预测模型,
该模型由序列编码器、状态解码器、并行样条插值模块和导数模块组成,能够以极低的存储空间消耗从连续时间域角度拟合复杂系统的动态过程。
实验环节利用膏体浓密机系统运行数据集对所述模型进行评估,可以发现该模型较好地克服了系统的非线性及长时延特性,在长期预测和短期预测场景中均表现出了较好的预测精度。

针对具有随机性的复杂动态系统建模问题,第四章提出了常微分方程-循环状态空间模型,该模型能够利用非均匀间隔采样的序列数据辨识具有随机性和长时滞特性的输入输出系统。
同时,该章提出了用于并行求解批常微分方程的参数化方法,使模型能够在时间点间隔不均匀的情况下并行、批量地推理与训练。同时该章提出的高效隐空间超调技术能够以极低的额外训练开销大幅度改善模型的多步预测性能。
实验环节采用三个具有随机性的输入输出系统数据集评估该模型的系统辨识效果,结果表明所提出模型在非均匀数据采样下获得比其他基线模型更好的预测效果,在高随机性、高缺失率以及测试集采样间隔分布改变的情况下优势更加明显。

针对复杂工业设备建模控制难的问题,第五章提出了基于连续时间有模型强化学习的自适应控制算法——启发式评价网络值迭代算法。
该方法可以在被控系统方程未知的情况下,利用系统实时输入输出数据实现控制策略的在线学习并获得较好的控制效果。
另外该章提出的短期经验回放技术能够有效提升评价网络参数的训练效率。
实验环节采用浓密机仿真模型对控制算法性能进行验证,结果表明相比其他自适应动态规划算法,所提出方法的网络收敛速度更快、在线追踪控制能力更优。且该算法成功部署应用于某矿场的浓密机控制系统中,使得底流浓度控制精度及稳定性优于原始规则控制策略。

针对具有周期多阶段转换特性的复杂动态系统建模问题,第六章提出了一种基于ODE-Net的自跳跃-常微分方程网络(AJ-ODE-Nets),同时采用该模型构建了用于长序列开环预测的编码器-解码器结构。
实验环节使用两个具有周期多阶段特性的系统运行数据集对所提出的AJ-ODE-Nets及预测框架进行评估。对于其中的膏体制备系统数据集,模型能够有效地预测螺旋给料器的启停点以及系统输出量。其中,水泥累积消耗量的预测误差小于2\%。
同时,该章基于预测仿真模型给出了最优的膏体浓度上限值以优化螺旋给料器的启停策略,有效减少单位体积充填成本8.2\%。

\section{未来工作展望}
% 采用微分方程网络能够从系统的非均匀采样数据中学习动态系统的运行模式及运行特征,基于该特征能够实现对于系统的预测与建模。
% 目前本论文已完成的工作均面向系统预测问题。未来计划从动态系统控制决策角度,结合微分方程网络在系统建模方面的优势,实现面向非均匀采样数据的强化学习方案。


% ODE-RSSM模型能够对不均匀流式序列数据进行编码,并在隐空间中生成连续时间的隐状态。
% % When the new monitored system inputs and outputs are fed, the GRU module updates the deterministic hidden state and infer new stochastic latent state.
% % Otherwise, the ODE-net predicts the evolution the $\b{h}_t$ and indirectly determines the distribution of $\b{z}_t$.
% 在同时给定新的输入输出数据时,ODE-RSSM采用GRU模块更新确定性隐状态,同时推理随机的隐变量,以此起到序列在线编码的作用。在仅有输入点时,模型能够使用ODE-Net预测$\b{h}_t$的变化并以此估计随机变量$\b{z}_t$的先验分布。
% % ODE-RSSM具备的在线编码、在线预测能力使得该模型在除系统预测以外的潜在应用。
% ODE-RSSM具备的在线编码、在线预测能力能够有效应用于除系统预测以外的问题及场景中。
% % On the basis of such property, this section will discuss the available applications of ODE-RSSM except for system prediction.

%  \textbf{模型预测控制}:在模型预测控制中(Model Prediction Control, MPC),ODE-RSSM可以作为优化过程所需的预测模型或仿真模型,
% 模型能够预测给定输入对系统序列输出的影响。ODE-RSSM支持状态循环更新的特性使其在处理增量数据时计算成本小,且优化序列长度范围灵活可变,在实际工业应用中具有很好的适用性。

% \textbf{异常检测:} 
% ODE-RSSM本质上为变分自编码器模型,利用其编码器和解码器模块能够估计给定序列数据的条件边缘似然。该似然能够作为监测数据是否存在异常的指标\cite{Xu2018}。

% 本节,我们分析了ODE-RSSM存在的限制以及不足,并以此延伸展望了未来可能的研究方向:

% \textbf{异步观测}
% ODE-RSSM假设所有通道的采样时间点是同步的。而在实际的工业系统中,不同传感器的数据采集是独立的,因此不同监测项的采集频率和采集时刻可能是不同的。
% 为了适用于此类异步观测数据,需要在模型中引入在线的数据连续化插值\cite{Che2018}技术。

% % \mathbf{Asynchronous observations} 
% % ODE-RSSM assumes the sampling time points of all channels are simultaneous. While different sensors may be observed at different frequencies in real industrial systems. 
% % It is worthwhile to incorporate online data imputation techniques\cite{Che2018} into the proposed framework.


% \textbf{时间演化过程中的随机性建模}:% ODE-RSSM is intrinsically a discrete-time model, which only infers the stochastic latent variables at discrete-time steps. 
% % The ODE-Net in CT domain only models the deterministic continuous-time evolution between adjacent observed time points.
% % In future, we look into complete continuous-time models with stochasticity, such as Stochastic Differential Equation Network(SDE), to model the continuous-time stochastic transitions of complicated system.

% 因为ODE-RSSM仅预测或推理出离散时间步下的随机隐变量分布,因此该模型本质上仍是一个离散时间模型。
% ODE-Net仅建模了相邻观测时间点之间确定性隐状态在连续时间域上的确定性演化,而无法建模状态演化过程中的存在随机性,如维纳过程等连续时间随机过程。
% 未来,本人将尝试引入具有随机性的连续时间模型作为ODE-NET的替代,如随机微分方程网络(Stochastic Differential Equation Neural Network SDE-Net),用于建模复杂系统中存在的连续时间随机性。
本文研究了面向复杂流程工业系统的建模方法,并提出了基于辨识模型的优化控制技术。尽管在各章实验环节主要介绍了上述方法在膏体充填场景的应用,但是本文从实际工业系统中抽象出的数学模型在众多领域是有通用性的。
% 其中的一些关键特性,如长时延、非确定性、不完全观测性等并非是充填系统特有的,对于许多其他的复杂场景同样适用。
因此,本节将对基于微分方程网络的建模及控制技术在其他领域的潜在应用进行展望。

\textbf{(1). 电力、能源、化工、矿业等流程工业场景下的系统建模与优化}:
包括本文着重探究的膏体充填过程在内,对于电力、能源、化工、矿业等流程工业场景,原料加工成为成品材料的物质转化过程是包含物理化学反应的气液固多相共存的连续化复杂过程,这一连续转化特性与本文关注的连续时间特性是一致的。另外,流程工业系统中广泛存在不同时间尺度、强非线性、多变量强耦合、生产条件变化频繁、随机性、长时延等特性,对于本文提出的基于微分方程与深度神经网络相结合的辨识模型,其模型结构与核心思想在不同工业系统中具有适用性。如何将本文提出的理论、技术在更多流程工业场景进行落地应用,如本文探究的充填系统一样,实现关键指标的预测与建模,并进一步从产品质量、物料成本角度实现优化,是笔者未来研究的重点方向。

\textbf{(2). 可微物理仿真技术}
物理仿真的经典范式是给定初值或边界条件,根据反映物理规律的微分方程进行推演,以模拟各种物理现象的演变过程。
近年来可微物理仿真技术成为了计算机图形学和机器人等领域的研究热点,并取得了广泛应用。作为经典物理仿真的延伸和拓展,可微物理仿真的核心思想是结合自动微分技术,将优化目标梯度信息沿时间反向回传,用于对物理系统中的材料参数、结构设计和控制信号等进行优化提升,以实现特定的设计或控制目标。在计算设计、材料参数估计等领域有着广泛的应用前景。
相比于可微物理仿真技术,本文提出的基于微分方程网络的系统建模等价于利用神经元架构描述隐空间中的系统状态变化,二者在微分方程求解、梯度传播、模型训练、建模目标等方面具有很强的相似性,在理论及应用方面的研究成果能够相互借鉴,推动彼此的发展。

\textbf{(3). 机械动力学系统建模与优化}:
以机器人作为典型代表的动力学系统具有典型的连续时间特性,且相比于流程工业系统,机械动力学系统具备观测空间更完备、外部噪音小等特点,该领域也是研究连续时间系统辨识的主要场景。传统机器人学中主要依靠机理分析的方式对关节点的加速度、阻尼、位移的动态关系进行建模。本文提出的连续时间域建模方法为解决机械动力学系统建模与优化提供了新的研究思路。同时,将动力学系统中已有的、可靠的机理先验与微分方程网络模型进行结合相对容易,部分学者已经围绕这一话题展开研究\cite{zhong2019symplectic,zhong2019symplectic},本节在接下来也将对如何实现二者结合这一问题进行讨论与展望。

另外,本文仅围绕常微分方程网络(ODE-Net)进行研究。由于ODE模型仅能描述系统对于时间的确定性导数,目前学界已经开始了其他微分方程系统与神经网络的结合研究,如本文第二章所介绍的随机微分方程网络、受控微分方程网络、以及偏微分方程网络(Partial differential equation network,PDE-Net)。如何利用上述模型的特异性以针对性地建模客观复杂系统,是未来值得深入研究的科研方向。

在数据驱动建模方法广泛应用之前,传统的基于理论科学及第一性原理的系统模型分析方法受限于基础理论发展的局限,难以适用于运行机理极为复杂的客观系统。
利用数据驱动算法对机理模型进行补充完善,可以有效提升机理模型对于不同应用场景的适应能力,一定程度上也减少了系统机理分析、模型参数估计的工作量。
从另一个角度看,向数据驱动模型注入已有机理模型的先验知识,能够提升数据驱动模型的可解释性,同时降低模型的自由度,减少训练数据需求量。另外,受机理先验的指导,模型在处理外推预测时的鲁棒性也会相应提升。

微分方程作为建模客观物理世界的一种基本工具,广泛存在于动力学、热力学、流体力学等复杂动态系统中。微分方程网络作为神经网路与微分方程的结合体,成功搭建了数据驱动算法与机理模型之间的桥梁,为解决数据科学问题带来了新的解决思路。笔者在此处给出三种微分方程与机理模型相互结合的方案,并给出未来研究展望。

\textbf{(1). 利用微分方程网络预测理论模型的动态参数}:
将理论机理模型应用于工程实践时,尽管其模型方程可以做到准确、可解释,
但是模型参数往往是难以获取的,且该参数很可能随时间发生演变。
利用微分方程网络的数据拟合与动态预测能力可以对机理模型中的未知参数进行建模。结合估计出的参数,再利用机理模型对目标系统进行预测分析。该方法既发挥了神经网络的可学习能力,也充分利用了机理模型的先验知识。
目前,该思想已被应用于可微物理建模领域中\cite{takahashi2021differentiable}。
不过,将神经网络与机理模型进行前后串联,使得网络的训练依赖于从机理模型传入来自于损失函数的梯度流。
如何克服由于机理模型复杂导致的参数不可导、网络难训练的问题
% ,以及如何建模混沌系统参数
是未来极有意义的研究方向。

\textbf{(2). 利用微分方程网络拟合机理模型的残差}:
一般情况下,利用机理模型可以拟合出被辨识过程的大致趋势,
% 而残差项往往受到超越当前认知的复杂因素的影响,想从模型机理角度出发克服这部分残差是极其困难的。
而一些超越当前科学认知的复杂因素往往会使得预测结果中带有一部分无法克服的残差项。
利用微分方程网络等黑盒模型对这部分“没有先验指导”的残差项进行建模是可行的。这种模式可以认为是机理模型(白盒)与神经网络模型(黑盒)的浅层结合。这种结合方式在方案实现、网络设计、参数训练等方面十分简单。不过该方法对于先验知识的使用没有自适应性,过于依赖机理模型的可靠性,对于过程机理难以准确构建、模型参数无法精确识别、机理先验不够完备的复杂系统,该方案很可能退化为纯黑盒网络模型,导致理论先验知识提供的价值无法被充分利用。

\textbf{(3). 利用机理模型指导微分方程网络设计}:
传统的微分方程网络模型多采用全连接网络结构,这种黑盒模型的参数训练结果完全由训练数据集决定,因此对数据量有极大的要求,且在处理外推预测时难以保证模型鲁棒性。利用机理模型中刻画的不同变量之间的相互关系或者约束关系,可以指导微分方程网络的设计,如Hamilton系统、拉格朗日系统等。不过现有的基于机理模型指导的网络设计方法都是定制化设计的,并没有形成通用的解决范式。对于机理公式及参数不完备、变量相关关系复杂的目标系统,如何设计更具有通用性的先验知识引入方法,是实现理论模型与微分方程网络深度结合的重要研究方向。
