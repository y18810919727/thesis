\chapter{总结}
\section{目前研究工作总结}
本文针对ODE动态提出了
本文围绕基于微分方程网络的动态系统建模及预测技术开展研究,针对复杂动态系统存在的非线性、长时延、多阶段周期转换等特性,探索了采用微分方程网络建模及预测系统动态的基础理论及关键技术。

针对复杂动态系统存在的非线性、长时延特性,本文提出了基于可微ODE-Net的高时延工业多输入输出系统预测模型,
该模型由顺序编码器、状态解码器、并行样条插值模块和导数模块组成,能够从连续时间域角度拟合复杂系统的动态过程。模型在长期预测和短期预测中均表现出了较好的性能。
利用膏体浓密机系统运行数据集对本文所述模型进行评估,可以发现模型较好地克服了系统的长时延特性,在长期预测和短期预测场景中均表现出了较好的预测精度。

针对复杂系统存在的周期多阶段转换特性,本文提出了一种基于ODE-Net的有限状态机-常微分方程网络(DFA-ODEnets),同时基于该网络构建了用于系统长序列预测的编码器-解码器结构。
该模型采用分层常微分神经网络(Hierarchical Neural ODE networks, H-ODEnet)拟合系统在不同阶段的动态过程,并引入时间预测器实现对系统所处阶段的自识别与自转移。
将本文提出的框架应用于具有典型的周期多阶段特性的制冷系统
实验环节,本文使用具有周期多阶段特性的数据中心制冷系统对所述的DFA-ODEnets及预测框架进行评估。经过实验验证,模型能够较好地预测包括阶段转换点在内的系统各个时刻的输出量。基于该预测框架能够实现对于系统制冷能耗的仿真,以及优化不同服务器负载下,最优的制冷系启动温度配置。

\section{下一步研究计划}
微分方程网络能够从系统的非均匀采样数据中提取动态系统的运行模式及运行特征,基于该特征能够实现对于系统的预测与建模。
目前本人已完成的工作均面向系统预测问题。未来计划从动态系统控制决策角度,结合微分方程网络在系统建模方面的优势,实现面向非均匀采样数据的强化学习方案。

