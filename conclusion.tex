\chapter{总结}
\section{本文研究工作总结}
% 本文针对ODE动态提出了
本文围绕基于微分方程网络的动态系统建模及预测技术开展研究,针对复杂动态系统存在的非线性、长时延、多阶段周期转换等特性,探索了采用微分方程网络建模及预测系统动态的基础理论及关键技术。

针对复杂动态系统存在的非线性、长时延特性,本文提出了基于可微ODE-Net的高时延工业多输入输出系统预测模型,
该模型由顺序编码器、状态解码器、并行样条插值模块和导数模块组成,能够从连续时间域角度拟合复杂系统的动态过程。模型在长期预测和短期预测中均表现出了较好的性能。
利用膏体浓密机系统运行数据集对本文所述模型进行评估,可以发现模型较好地克服了系统的长时延特性,在长期预测和短期预测场景中均表现出了较好的预测精度。

针对具有周期多阶段转换特性的复杂动态系统建模问题,本文提出了一种基于ODE-Net的自跳跃-常微分方程网络(AJ-ODEnets),同时采用该模型构建了用于长序列开环预测的编码器-解码器结构。
实验环节使用具有周期多阶段特性的制冷系统运行数据对所提出的AJ-ODEnets及预测框架进行评估,验证了该模型对于阶段转换点以及系统输出量预测的有效性,功耗预测相对误差小于5\%。同时,该章基于预测框架能够实现对于系统制冷能耗的仿真,对于不同热负载给出了最优的制冷系启动温度配置,有效减少制冷能耗6-25\%。

针对具有随机非确定性的复杂动态系统建模问题,本文提出了常微分方程-循环状态空间模型,该模型能够利用非均匀间隔采样的序列数据识别具有随机非确定性和长时滞特性的输入输出系统。
同时,本文提出了用于并行求解批常微分方程的参数化方法,使模型能够在时间点间隔不均匀的情况下并行、批量地进行推理与训练。
实验环节采用三个具有随机非确定性的输入输出系统数据集评估该模型的建模效果,结果表明所提出模型在不规则数据采样下相比基线模型具有更好的预测准确性,在高缺失率以及测试集采样间隔分布改变的情况下优势更加明显。

针对复杂工业设备建模控制难的问题,本文提出了基于有模型强化学习的自适应控制算法——启发式评价网络值迭代算法。
该方法可以在被控系统方程未知的情况下,仅利用系统实时输入输出数据实现控制策略的在线学习并获得较好的控制效果。
另外本文提出了短期经验回放技术以增强评价网络训练的稳定性。
实验环节采用尾矿浓密机仿真模型对控制算法性能进行验证,结果表明相比其他自适应动态规划算法,所提出方法的网络收敛速度更快,在线追踪控制能力更优。且该算法成功部署应用于某矿场的浓密机控制系统中,使得底流浓度控制误差更低、稳定性更优。


\section{未来工作展望}
基于微分方程网络能够从系统的非均匀采样数据中提取动态系统的运行模式及运行特征,基于该特征能够实现对于系统的预测与建模。
目前本论文已完成的工作均面向系统预测问题。未来计划从动态系统控制决策角度,结合微分方程网络在系统建模方面的优势,实现面向非均匀采样数据的强化学习方案。


% ODE-RSSM模型能够对不均匀流式序列数据进行编码,并在隐空间中生成连续时间的隐状态。
% % When the new monitored system inputs and outputs are fed, the GRU module updates the deterministic hidden state and infer new stochastic latent state.
% % Otherwise, the ODE-net predicts the evolution the $\b{h}_t$ and indirectly determines the distribution of $\b{z}_t$.
% 在同时给定新的输入输出数据时,ODE-RSSM采用GRU模块更新确定性隐状态,同时推理随机的隐变量,以此起到序列在线编码的作用。在仅有输入点时,模型能够使用ODE-Net预测$\b{h}_t$的变化并以此估计随机变量$\b{z}_t$的先验分布。
% % ODE-RSSM具备的在线编码、在线预测能力使得该模型在除系统预测以外的潜在应用。
% ODE-RSSM具备的在线编码、在线预测能力能够有效应用于除系统预测以外的问题及场景中。
% % On the basis of such property, this section will discuss the available applications of ODE-RSSM except for system prediction.

%  \textbf{模型预测控制}:在模型预测控制中(Model Prediction Control, MPC),ODE-RSSM可以作为优化过程所需的预测模型或仿真模型,
% 模型能够预测给定输入对系统序列输出的影响。ODE-RSSM支持状态循环更新的特性使其在处理增量数据时计算成本小,且优化序列长度范围灵活可变,在实际工业应用中具有很好的适用性。

% \textbf{异常检测:} 
% ODE-RSSM本质上为变分自编码器模型,利用其编码器和解码器模块能够估计给定序列数据的条件边缘似然。该似然能够作为监测数据是否存在异常的指标\cite{Xu2018}。

% 本节,我们分析了ODE-RSSM存在的限制以及不足,并以此延伸展望了未来可能的研究方向:

% \textbf{异步观测}
% ODE-RSSM假设所有通道的采样时间点是同步的。而在实际的工业系统中,不同传感器的数据采集是独立的,因此不同监测项的采集频率和采集时刻可能是不同的。
% 为了适用于此类异步观测数据,需要在模型中引入在线的数据连续化插值\cite{Che2018}技术。

% % \mathbf{Asynchronous observations} 
% % ODE-RSSM assumes the sampling time points of all channels are simultaneous. While different sensors may be observed at different frequencies in real industrial systems. 
% % It is worthwhile to incorporate online data imputation techniques\cite{Che2018} into the proposed framework.


% \textbf{时间演化过程中的随机性建模}:% ODE-RSSM is intrinsically a discrete-time model, which only infers the stochastic latent variables at discrete-time steps. 
% % The ODE-Net in CT domain only models the deterministic continuous-time evolution between adjacent observed time points.
% % In future, we look into complete continuous-time models with stochasticity, such as Stochastic Differential Equation Network(SDE), to model the continuous-time stochastic transitions of complicated system.

% 因为ODE-RSSM仅预测或推理出离散时间步下的随机隐变量分布,因此该模型本质上仍是一个离散时间模型。
% ODE-Net仅建模了相邻观测时间点之间确定性隐状态在连续时间域上的确定性演化,而无法建模状态演化过程中的存在随机性,如维纳过程等连续时间随机过程。
% 未来,本人将尝试引入具有随机性的连续时间模型作为ODE-NET的替代,如随机微分方程网络(Stochastic Differential Equation Neural Network SDE-Net),用于建模复杂系统中存在的连续时间随机性。




传统的基于理论科学及第一性原理的系统模型分析方法受限于基础理论发展的局限,难以适用于运行机理极为复杂的客观系统。
利用数据驱动算法对机理模型进行补充完善,可以有效提升机理模型对于不同应用场景的适应能力,一定程度上也减少了机理分析、参数估计的工作量。
从另一个角度看,向数据驱动模型注入已有机理模型的先验知识,能够提升数据驱动模型的可解释性,同时降低模型的自由度,减少训练数据需求量。另外,受机理先验的指导,模型在处理外推预测时的鲁棒性也会相应提升。

微分方程作为建模客观物理世界的一种基本工具,广泛存在于动力学、热力学、流体力学等复杂动态系统中。微分方程网络作为神经网路与微分方程的结合体,能够有效地搭建数据驱动算法与机理模型之间的桥梁,为解决数据科学问题带来新的解决思路。笔者在此处给出三种微分方程与机理模型可以相互结合的方案,并给出未来研究展望。

\textbf{利用微分方程网络预测理论模型的动态参数}
将理论机理模型应用于工程实践时,尽管其模型方程可以做到准确、可解释,
但是模型参数往往是难以获取的,且该参数很可能随时间发生演化。
利用微分方程网络的数据拟合与动态预测能力可以对机理模型中的未知参数进行建模。根据参数估计结果,再利用机理模型对目标系统进行预测分析。该方法既发挥了神经网络的可学习能力,也充分利用了机理模型的先验知识。
目前,该思想已被应用于可微物理建模领域中\cite{takahashi2021differentiable}。
不过,将神经网络与机理模型进行前后串联,使得网络的训练依赖于损失函数从机理模型传来的梯度流。
如何克服机理模型复杂导致的参数不可导、网络难训练的问题,以及如何建模混沌系统参数是未来极有意义的研究方向。

\textbf{利用微分方程网络拟合机理模型的残差}
一般情况下,利用机理模型可以拟合出被辨识过程的大致趋势,
% 而残差项往往受到超越当前认知的复杂因素的影响,想从模型机理角度出发克服这部分残差是极其困难的。
而一些超越当前科学认知的复杂因素往往会使得预测结果中带有一部分无法克服的残差项。
利用微分方程网络等黑盒模型对这部分“没有先验指导”的残差项进行建模是可行的。这种模式可以认为是机理模型(白盒)与神经网络模型(黑盒)的浅层结合。这种结合方式在方案实现、网络设计、参数训练等方面十分简单。不过该方法对于先验知识的使用没有自适应性,过于依赖机理模型的可靠性,对于过程机理难以准确构建、模型参数无法精确识别的复杂系统,该方案很可能退化为纯黑盒网络模型,导致理论先验知识提供的价值无法充分利用。

\textbf{利用机理模型指导微分方程网络设计}
传统的微分方程网络模型多采用全连接网络结构,这种黑盒模型的参数训练结果完全由训练数据集决定,因此对数据量有极大的要求,且在处理外推预测时难以保证模型鲁棒性。利用机理模型中刻画的不同变量之间的相互关系或者约束关系,可以指导微分方程网络的设计,如Hamilton系统、拉格朗日系统等。不过现有的基于机理模型指导的网络设计方法都是定制化设计的,并没有形成通用的解决范式。对于机理公式及参数不完备、变量相关关系复杂的目标系统,如何设计更具有通用性的先验知识引入方法,是实现理论模型与微分方程网络深度结合的重要研究方向。

