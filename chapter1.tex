% !Mode:: "TeX:UTF-8"

\chapter{绪论}

\pagenumbering{arabic}
\setcounter{page}{1}
在工业场景下,动态系统建模在过程控制、状态估计、系统预测等众多领域都起到了举足轻重的支撑作用。由于现实世界大部分的动态系统具有非线性、高扰动、强耦合等复杂特性,从机理角度建模系统动态过程的传统分析方法难以满足实际应用要求。伴随着工业监测技术的不断完善以及生产自动化、信息化水平的不断提升,各种大型设备及生产过程均安装了用于实时监测数据的传感器。 由于监测数据获取成本低廉,且现有辨识理论及方法存在一定局限性,使得基于数据驱动的复杂工业系统建模方法广受学者们的关注。然而,系统建模问题本身的高复杂性以及被辨识系统的多样性使得模型的辨识效果对于模型选择与设计极其敏感。如何根据目标系统的不同特性,设计合理的参数化模型以获得最佳的建模效果,是领域内亟待解决的关键问题。本文分别面向具有长时延非线性、周期多阶段性、随机非确定性的三种复杂工业系统,提出了以常微分方程网络为骨架的三种模型架构,有效实现了以系统先验为指导、以离线数据为原料、以神经网络为骨架,端到端建模复杂动态系统的目标。同时,
并在识别模型的基础上,提出了基于有模型强化学习理论的在线控制优化方法,并成功应用于工业实践,搭建了系统建模与控制决策之间的桥梁。
\section{研究背景与意义}
    
% 当前,新一轮科技革命和产业变革加速发展,新一代信息技术正在与工业生产 深度融合,自动化、信息化、智能化已经成为全球工业制造发展的重要方向。
% 2020年李克强总理发布的《政府工作报告》中,明确提出“推动制造业升级”、“推进智能制造”的要求。如何工业生产利用智能化技术,将,

% 为了实现工业4.0、工业互联网、数字孪生等伟大技术愿景,其中


% 复杂工业系统的控制与优化是典型的交叉学科,计算机、自动化、控制工程、仪器仪表等。
% 伴随着人工智能与数据科学的发展,数据驱动的复杂系统建模与控制优化技术逐渐应用。
% 现存问题:如何合理地将深度学习与工业过程特性深度融合,同时有效地指导系统优化并实现智能控制,对此部分的研究迫在眉睫。
% 本文面向具有长时延特性、周期多阶段性、非确定性的三类复杂工业系统,提出一套基于ODE-Net的动态系统建模方法,并在构建模型的基础上,提出了基于有模型强化学习理论的控制优化方法,并成功应用于工业实践。实现了xxx目标,为数据驱动的工业系统建模及控制优化提供了新的研究思路。

\subsection{动态系统辨识及有模型控制的起源与发展}

% 基于数据驱动的复杂工业系统预测建模方法可以分成两大类:离散时间(Discrete Time, DT)系统建模和连续时间系统建模。
复杂系统及设备的分析与优化技术涉及了集工业制造、自动控制、计算机、人工智能等多学科知识,长久以来受到了国内外学者的广泛关注与深度研究。
如何建模复杂系统的运行过程并认知系统的内在机理,是实现系统分析优化的基础。
想要解决复杂系统的建模问题,其本质在于对系统关键变量的自相关趋势以及受其他协变量的影响进行建模。
由于在工业环境下,复杂系统的运行过程存在非线性、高扰动、强耦合等复杂特性,从物理、化学、动力学角度建模系统动态过程的传统分析方法不再适用。伴随着工业监测技术的不断完善以及生产自动化、信息化水平的不断提升,各种大型设备上安装了用于实时监测生产数据的传感器。低廉的数据获取成本以及理论建模的局限性使得数据驱动方法成为建模复杂工业系统过程的主流方案。

最早追溯到1950年,为了进行控制系统设计,文献\cite{zadeh1956identification}首次提出了系统识别的概念用来建模动态系统。
其核心目标是寻找一个与系统“相符”的模型,使得模型预测的输出尽量接近给定真实的系统输出。
以数据为核心驱动力的传统系统识别方法已经发展为一个非常完善且成熟的研究领域。
\cite{le2013system,gevers2006personal,ljung2008perspectives,ljung2011four,Ljung2020}。

在《关于发布未来工业互联网基础理论与关键技术重大研究计划2021年度项目指南的通告》中指出,“实现动态扰动下系统分布式资源调控、数据驱动的系统建模、质量预测与控制以及全流程重构的多目标优化,结合航空航天$\cdots\cdots$。”足以说明数据驱动系统建模技术在现代工业智能化发展路线中占据着举足轻重的地位。

AI界当代最著名巨擘之一、Meta的AI实验室灵魂人物Yann LeCun,长期致力于让机器对世界的运转理念有基础了解。在其设计的通用人工智能(Artificial General Intelligence,AGI)架构体系中,设计了配置器、短期记忆模块、感知器、决策器、世界模型、评价模型,其中世界模型放在了与感知器和决策器同等重要的位置上。理想的世界模型能够像拥有“常识”的人类一样,预见给定行为后将产生后果,并辅助智能体决策。本文探究的动态系统建模可以认为是世界模型在低维控制、低维观测限定下的特例。将单个输入输出控制系统的观测、感知、建模预测、决策定义为一个小的世界。与通用人工智能的研究在范式层面具有较强的相似性。

从机器学习的视角来看,系统建模本质上是一种有监督学习问题\cite{jordan1992forward}。本节
% TODO

\subsection{微分方程网络在复杂系统建模中的应用}

依托于近年来机器学习、强化学习\cite{sutton2018reinforcement}、深度学习\cite{lecun2015deep}\cite{duan2016}、时间序列分析技术\cite{shumway2000time}的发展与普及,从离散时间角度的建模研究方向涌现出了诸多成果,领域发展相对较为成熟,其原因与数字计算机将信息离散化的思想密不可分。相比而言,连续时间系统的相关理论体系研究较为滞后,且在当今大数据、数据驱动的时代背景下,并未与主流技术形成深度融合。但是对比离散时间系统建模方法,从连续时间域角度进行系统建模的思想在某些情况下具有较大优势:
\begin{itemize}
\item	对于物理属性的兼容性:大部分的物理现象都服从连续时间设定,因此利用连续时间模型对物理系统进行建模可以更好地匹配客观世界物理规律,且增加模型的可解释性。
\item	对于先验知识的表示:系统动态不同阶的相关度,如:位移、速度、加速度之间的相关性可以在连续时间系统下非常容易地表示出来,进而有效利用先验知识,降低模型求解的自由度。
\item	数据滤波:连续时间系统建模在处理有噪数据时具有天然的优势,而对于离散时间系统建模方法来说,去噪过程和建模过程完全分开独立的。
\item	非均匀的数据采样:当数据采样间隔不均匀时,离散时间方法无法使用,但是连续时间方法不受限于此。
\item	CT系统与DT系统的相互转换:对于CT系统可以通过变更采样率的方法获得任意时间间隔的DT系统,而DT系统则不然。当数据采样周期与下游应用所需的时间间隔不一致时,DT系统建模方法难以适用
\item	DT模型在高采样率下存在精确性问题:当数据采样率较高时,系统状态变化过程接近环形,不适宜的DT模型很容易存在精度差、鲁棒性差的问题;
\item	钢性系统:当系统同时存在快过程和慢过程时,离散采样的间隔选取不当很容易造成快过程或慢过程其中一类统计性质的丢失,对离散时间系统处理造成较大阻碍。
\end{itemize}
% \subsection{基于数据驱动的复杂工业系统建模及控制}
系统通常是由表征系统输入输出关系的数学模型描述的,这个模型有其特定的结构和参数。以代数方程表示的系统称为静态系统,
考虑最简单的形式,系统的连续时间模型可以建模为常系数微分方程:
\begin{equation}
\frac{\mathrm{d}^{n} y(t)}{\mathrm{d} t^{n}}+a_{1} \frac{\mathrm{d}^{n-1} y(t)}{\mathrm{d} t^{n-1}}+\cdots+a_{n} y(t)=b_{0} \frac{\mathrm{d}^{m} u(t)}{\mathrm{d} t^{m}}+\cdots+b_{m} u(t)+v(t)
\end{equation}
$\frac{d^{i} x(t)}{d t^{i}}$代表信号$x(t)$对时间的 阶导数,该式表示了在任一时刻系统状态对时间的各阶导数与输入量对时间的各阶导数之间存在线性关系。对上式进行简化,仅考虑状态的最高一阶导数与输入量的零阶导数,并引入非线性成分,得到:
\begin{equation}
\dot {\boldsymbol y}(t)=f(\boldsymbol y(t), \boldsymbol{u}(t))
\end{equation}
其中 为某一未知函数且具有较强的非线性特性,相比于线性系统没有通用的非线性系统参数识别方法。神经网络近似能力的通用性可以作为建模非线性系统的一种手段\cite{funahashi1993approximation}。利用神经网路的前向传播过程模型系统动态过程\cite{temeng1995model, tan1996nonlinear}。
然而此类方法的缺陷的只能提供有限步,且一般情况是单步的预测,而无法起到系统识别与仿真的作用。循环神经网络(Recurrent neural network, RNN)因为存在隐状态,可以更好地处理长期预测问题并对系统进行建模\cite{delgado1995dynamic, zamarreno1998state}。
处理连续时间系统识别问题时,一个典型的困难是采样间隔分布不一致的问题。一种针对线性系统的传统解决方案是利用状态变量滤波方法,该方法可以近似给出各采样点对时间的各阶导数。非线性情形下,Demeester等\cite{demeester2020system}采用时间变化的神经序列模型解决该问题。

% \subsection{基于编解码结构的系统预测}
% 当系统具有较大时延,即高时滞特性下,模型需要考虑历史数据对未来系统变化的影响。Felipe 等\cite{demeester2020system}利用带有Attention组件的Encoder-Decoder模型对一种名为膏体浓密机系统进行建模识别,Yuan等\cite{Yuan2020}采用一种更复杂的名为双注意力循环神经网络(Dual Attention Recurrent Neural Network, DARNN)的RNN网络对工业系统进行建模,两种方法都考虑了系统变量之间的长期依赖性,利用循环神经网络加Attention机制的强大编码能力,对历史数据、不同维度数据进行信息编码,来辅助输出量的预测,并且利用预测误差对编码器部分进行训练调整。然而两种方法都是针对离散时间系统设计的,不适用于连续时间系统建模问题。
% \section{深度微分方程网络}
% \subsection{深度微分方程网络总描述}
% \subsection{微分方程网络概述}
发表自Nip2018的一篇开创性文章\cite{chen2018neural}提出了一种常微分方程神经网络,该网络可以模拟常微分的计算过程,并且给出了一种内存复杂度为常数量级的网络训练方法,网络参数的梯度求解等价于求解另一个常微分方程。
ODE-Net的主要应用包括Res-Net的替代、时间序列建模以及可逆正则化流\cite{Grathwohl2019}。本文主要围绕ODE-Net在时间序列建模中的应用展开研究。
因为ODE-Net的连续时间特性,该方法能够将深度网络模型应用于连续时间域下的时间序列建模问题研究中,开辟了时间序列分析、连续时间系统建模研究的新思路。
% 接下来本节将简单介绍几种常见的微分方程网络及其应用。

% \subsection{常微分方程网络}
Chen等最早提出常微分方程网络化\cite{chen2018neural}开创了深度学习与微分方程结合的先河。
为了加速ODE-Net的求解,
Zhuang等\cite{Zhuang2020}提出了自适应检查点联合状态方法以改进原始adjont方法求解梯度的精度以及效率。正则化神经常微分方程(regularized neural ODE,RN-ODE)\cite{J2020}基于理论保证的最优传输和稳定性正则化简化ODE系统,能够有效优化ODE-Net的数值积分求解效率。
除此之外,对ODE方程的高阶导数正则化\cite{kelly2020}对求解时间点添加随机扰动\cite{Ghosh2020}等方案也能够起到正则化作用,加速ODE-Net的求解效率。

在处理时间序列问题时,相比于RNN等离散时间步下的循环神经网络。ODE-Net天然地适用于建模非均匀采样的时间序列。如基于ODE-Net衍生的ODE-RNN\cite{10.5555/3454287.3454765}、GRU-ODE-Bayes\cite{brouwer2019gru}、ODE-LSTM\cite{lechner2020learning}。
为了求解ODE-Net的初值,Latent ODE\cite{10.5555/3454287.3454765}将ODE-Net的初始状态视为先验分布为标准高斯分布的因变量,利用编码器-解码器结构实现序列的重构、插值、预测,并使用变分贝叶斯方法对模型进行训练。

文献\cite{Yildiz2019}基于贝叶斯理论,构建了2阶ODE-Net模型用于高维时间序列的建模,同时利用ODE-Net连续正则化流的特性估计待预测时间点隐变量的后验分布并引入分布正则化对隐状态的范围作出限定。

% \subsection{受控微分方程网络}
由于常微分方程的解仅由初始状态决定,无法基于后续的观测对轨迹进行调节,使得诸如latent-ode、ODE2VAE等模型将序列观测信息全部压缩至初始状态中,有可能造成观测信息的信息损失。
神经受控微分方程(Neural Controlled Differential Equation)\cite{kidger2020neural}将受控信号的微分项融入在ODE网络的求解中。相比于在求解时间区间的间隔点处利用观测数据对隐状态进行更新的方法(如ODE-RNN、ODE-LSTM等),获得了更好的序列信息提取与表示能力。
为了解决神经受控微分方程在长时间序列场景下难以训练的问题,基于对数签名变换(Log signature transform)的Log-signature/NCDE方法\cite{morrill2021neural}通过签名变换对受控信号进行转换,能够增加模型训练速度、减少存储开销、改进模型预测性能。
针对常规的受控微分方程需要对离散序列进行样条插值,无法实现在线预测的情况,Morrill等\cite{morrill2021online}描述了受控微分方程中连续控制信号应满足的性质,并给出了三次埃尔米特直线插值方法,使受控微分方程网络能够类似于RNN一样,实现在线的时间序列数据处理,而不需要预先对完整的序列数据进行插值。
% \subsection{随机微分方程网络}

随机微分方程相比于常微分方程项添加了扩散项。
对于随机微分方程的训练也可以采用联合敏感度法(Adjoint Sensitibity)进行训练\cite{li2020scalable}。由于随机微分方程网络的前向传播过程依赖于带有随机性的Wiener过程,为了避免存储完整的计算图,需要保证反向求解SDE时对Wiener过程的采样与前向传播保持一致。文献\cite{li2020scalable}利用基于虚拟布朗树(Virtual Brownian Tree)的伪随机数生成策略,仅需常数级存储$\mathcal{O}(1)$即可对的SDE网络前向传播的Wiener过程采样结果进行存储。节约存储空间的代价为对特定时间点下的wiener过程采样时间复杂度为$\mathcal{O}(\log n)$。
将微分方程网络作为Res-Net的替代品处理图像领域问题时,有研究表明将ODE-Net替代为SDE-Net,在ODE-Net中增加随机扩散项,也能够起到随机正则化的作用,提升网络的泛化能力\cite{Oganesyan2020}。
神经跳变随机微分方程(Neural Jump Stochastic Differential Equations,NJSDE)\cite{Jia2019}将扩散项中对时间的微分替换为观测点次数的微分,能够在建模系统隐空间连续时间动态的同时对观测点出现事件本身以及时刻进行建模,该模型有效地应用于地震预测及药物预测。

% \subsection{受控微分方程网络}
% \section{深度微分方程网络的应用}
% \subsection{}
ODE-Net作为Res-Net的替代,可以应用于图像处理问题中:如图像超分问题\cite{OISR,jia2019focnet}。
在系统辨识领域,
Zhong等\cite{zhong2019symplectic}采用ODE-Net对符合哈密尔顿特性的动态系统进行建模学习,巧妙地将物理先验知识融入到学习模型的设计中。并有效地应用于符合哈密尔顿性质的刚体系统建模与控制问题中。
Ayed等\cite{ayed2019learning}采用ODE-Net模型从系统状态的部分可观测信息学习复杂非线性时空过程。该方法有效应用于水流动预测、Navier Stokes方程、海洋温度分析。
SNODE\cite{Quaglino2019}模型基于勒让德多项式构建ODE-Nets的压缩表示并应用于系统辨识。

为了建模来自于多智能体系统的非均匀采样数据,LG-ODE模型\cite{Huang2020}将图神经网络、自注意力以及ODE-Net进行结合,利用时序自注意力模型构建微分方程求解的初态,采用图神经网络建模观测点的时序依赖关系和不同观测项之间的空间依赖关系,并以此为基础估计常微分方程中的隐状态导数。模型有效地应用于稀疏序列的插值预测与外推预测。


对于Transformer、AttentionSeq2Seq等基于注意力的序列处理模型,利用ODE-Net的连续时间特性可以构建连续时间注意力模型\cite{chen2021continuous},并将注意力机制应用于非均匀采样的时间序列。
近年来,Transformer模型\cite{Vaswani2017}因在序列数据处理问题上的优异表现受到了学者的广泛关注,基于Transformer的长序列预测模型,如Informer\cite{Zhou2020}、Autoformer\cite{Wu2021}等均获得比传统时序预测算法更优的预测精度。
得益于ODE-Net的连续时间特性,该模型能够与Transformer进行结合,处理非均匀采样数据或长序列预测及建模问题,如构建连续时间注意力\cite{chen2021continuous}、辅助位置编码(position embedding)\cite{Liu2020}。

ODE-Net模型能够利用非均匀采样数据拟合动态系统,因此可用于有模型强化学习\cite{Yildiz2021}领域,构建可微分的系统状态演化估计器,进而辅助策略模块的学习。
对于系统动态已知的连续时间系统,利用ODE-Net可以为策略网络构建连续时间梯度估计器\cite{Ainsworth2020},使控制和仿真任务的学习更高效、更鲁棒。



\section{本文研究的关键问题}
\label{sec:challenge}
现有的基于机器学习的系统预测方法多从离散时间域角度描述系统动态过程,并利用数据驱动的方式对离散时间系统参数进行拟合训练。
但在复杂工业系统中,上述陈列的系统特性以及问题需求是时常存在的。
比如由于不同传感器工作频率不一致会导致数据存在非均匀采样情况,使用离散时间域模型之前需要对数据进行大量的前处理,这会对数据的原始特性造成损坏。

连续时间域模型对于复杂动态系统具有天然契合性,深度学习方法在参数化建模与复杂系统表示方面具有较大优势。
因此,本论文从二者结合的角度,对基于深度微分方程网络的复杂动态系统预测技术开展研究。
利用参数化的深度神经网络模型拟合复杂系统的微分方程,并基于拟合模型实现系统的预测、控制与优化。
使用连续时间域的深度神经网络模型拟合复杂系统,会面临以下研究难点与挑战:
\subsection{具有长时延、强非线性的复杂连续时间系统建模}
传统的连续时间系统辨识方法主要利用线性模型拟合系统的高阶微分方程。然而,这种方法不适用于建模具有长时延、高噪音、强非线性的不完全观测系统。
与此同时,真实工业系统的运行过程往往遵循各种物理化学定律,对于系统的演化过程用连续时间模型进行建模更加契合。而非线性连续时间模型的求解依赖于近似数值积分算法,这对于依托深度神经网络的,无疑限制了模型的大小和训练效率。
当使用深度神经网络拟合系统的连续时间动态时,如何有针对性地对网络进行设计以适用于上述复杂系统,是一个尚待解决的技术难题。另一方面,复杂微分方程的求解依赖于数值近似方法,数值近似求解过程中的大量计算使训练深度神经网络所需的计算图过于庞大。如何在保证数值求解精度的同时降低显存占用,也是将该理论方法应用于工程实践时亟待解决的关键问题。

\subsection{具有周期多阶段跳变性的复杂连续时间系统建模}
部分复杂工业系统在运行时呈现周期性多阶段特点,阶段的持续时间受多变量影响且不同阶段内呈现的系统动态特性具有极大差异。使用单一模型难以准确地拟合系统的所有阶段。另一方面,多输出系统中不同输出变量的时序特性存在较大差异,如何通过调整网络模型结构以对不同变量进行差异化处理也是亟待解决的研究问题。

\subsection{具有随机性、非确定性的复杂连续时间系统建模}
现存的连续时间系统辨识及动态系统建模方法,如Time-Aware RNN\cite{Demeester2019}、SNODE\cite{Quaglino2019} 仅在确定性状态空间对模型进行表示。首先,确定性模型没有引入任何随机性成分,不便于实现蒙特卡洛采样,这使得某些基于随机采样预测的控制规划方法,如交叉熵(Cross entropy Maximum,CEM)、蒙特卡洛树搜索(Monte Carlo Tree Search,MCTS)难以与此类系统建模方法进行配合使用。
另外,确定性模型显然无法适用于建模带有随机性的模型,当被辨识系统的状态转移过程本身具备较强的随机性时,可以近似认为状态之间的转移过程服从某种复杂的分布,理想的辩识模型应该能够直接对该分布进行建模,如离散时间域下的循环状态空间模型(Recurrent State Space Model, RSSM)。
最后,确定性模型无法对系统当前状态的非确定性进行度量与表示。因为在现实世界中,尽管很多系统的转移过程本身是确定的,但由于其观测空间的不完备性,从可观测的输入输出数据中无法准确推理系统的内部状态,可以近似认为系统是存在非确定性的。现有连续时间域的系统辨识方法仅能在隐空间中隐式地对系统的非确定性进行编码,而无法对其显示地量化与评估,制约了模型的可解释性与拟合能力。

因此本文着重研究存在随机性、非确定性的复杂系统连续时间域系统建模方法,使辨识模型能够对系统的随机特性进行拟合,并在给定观测数据下评估系统的非确定性。

\subsection{面向复杂工业系统的控制策略构建与自适应优化}
工业场景下,大部分复杂生产过程往往伴随着较强的非线性、非确定性、高时滞性,因此难以建立准确的数学模型近似其运转机理, 导致传统的控制优化方法无法适用于此类复杂工业设备。
目前业界对基于强化学习理论的最优控制技术\cite{Sutton2018}\cite{F.L.LewisD.Vrabie2012}寄予厚望,希望能够以免模型、数据驱动的方式实现复杂工业系统的自适应优化控制。
然而,受限于高采样成本与低场景泛化能力,无模型强化学习算法在真实的工业实践中难以适用。
同时考虑到工业过程试错成本高,大部分强化学习算法随机设定策略模型的参数,难以在模型训练初始阶段,保证生产过程的安全。

因此本文研究基于模型的复杂工业系统优化控制策略,充分利用系统运行时的离线数据构建预测模型,并在辨识模型的基础上构建具有在线自学习能力的控制决策模型,该方法能够适应物料性质改变、设备老化等被控系统不断变化的情况。

\section{本文的研究内容}
% 针对上述问题与挑战, 本文以具有连绔时间动态特性的复杂系统作为研究对 象,针对系统存在的非线性、非确定性、多阶段混合、高时延、不同输出量统计特 征不一致等特性, 将连绔时间域模型的灵活性与深度神经网络的强大表示能力相
% 针对第\ref{sec:challenge}节提出的系统存在的高时延、非线性、随机非确定性、周期多阶段性、不同输出量统计特征不一致等复杂特性特性,
针对第\ref{sec:challenge}节提出的复杂工业系统难以建模预测及优化控制的问题。
% 本文以具有连时间动态特性的复杂系统作为研究对 象,针对系统存在的非线性、非确定性、多阶段混合、高时延、不同输出量统计特 征不一致等特性, 将连绔时间域模型的
本文以具有连续时间动态特性的复杂系统作为研究对象,
针对系统存在的非线性、非确定性、多阶段混合、高时延、不同输出量统计特征不一致等特性,
依托于连续时间域模型的灵活性与深度神经网络的强大表示能力,研究基于连续时间深度时序网络的系统建模方法。
并在识别模型的基础上,研究基于有模型强化学习理论的在线优化控制方法,并应用于工业实践。
% 针对复杂系统存在的非线性、非确定性、多阶段混合、高时延、不同输出量统计特征不一致等特性,本文开展了如下几方面的研究:
本文研究内容的具体介绍如下:

\subsection{基于可微ODE-Net的高时延复杂工业系统预测}
第一项研究内容基于可微ODE-Net的高时延复杂工业系统预测,重点讨论了连续时间视角下,非线性、高时延系统的系统建模问题。
大部分复杂工业系统反馈延迟长,现有辨识方法难以拟合长距离的依赖关系,如果从状态空间模型的角度将系统状态和输入记忆进行压缩,会对待学习过程引入了极强的复杂性与非线性。
另外基于数据驱动的自回归系统辨识模型受制于预测过程累积误差的存在,难以同时处理系统短期预测和长期预测问题。
因此需要从探索\textbf{参数化微分方程网络对于客观世界连续时间过程的可拟合性}这一科学问题出发,研究深度学习网络与连续时间过程相结合的系统建模方法。

本文的第一个研究内容为了克服动态系统存在的非线性、非仿射的高复杂性。研究了基于深度学习的复杂工业系统建模方法,同时为了契合生产设备运转本质为连续时间过程的特性,提出以ODE-net作为模型骨架结构,从连续时间域角度拟合复杂工业系统的动态过程。
针对系统输入输出之间长时延依赖关系难以建模的问题,研究了基于序列自编码器模型的输入输出长距离信息连接通道构建方法。
针对深度序列模型受制于累积预测误差,难以有效进行长期预测的问题,基于时间序列稳定性理论,分析了普通常微分方程网络在长期预测中的非稳定特性,并提出两种分别适用于短期预测和长期预测任务的导数模块定义方法。同时,针对ODENet网络训练需要连续控制输入信号的问题,研究了面向训练过程的离散输入点并行插值方法。最后,提出了一套由序列编码器、状态解码器和导数模块组成的深度连续时间(Continuous Time, CT)系统辨识网络模型,能够以端到端的方式学习工业系统输出的自回归变化过程和输入对输出的非线性影响。
通过消融实验探究了编码器输入序列长度、微分方程求解器选择对于预测精度的影响。最后,通过某真实铜矿场中膏体浓密机的运行数据验证了本文提出方法在解决高时延、非线性复杂系统长短期预测问题的有效性。


% 针对复杂工业系统本质上为连续时间演化过程,且存在非线性、长时延等特性,本文提出以ODE-net作为模型骨架结构,从连续时间域角度拟合复杂工业系统的动态过程。


\subsection{基于自跳跃-常微分方程网络的连续时间周期性跳变系统建模}


% 2) 基于有限状态机-常微分方程网络的周期性多阶段复杂系统建模 \\
第二项研究内容基于自跳跃-常微分方程网络的连续时间周期性跳变系统建模重点讨论了连续时间视角下周期跳变系统建模问题。
工业场景中,系统在不同阶段下的动态特性彼此迥异,且各阶段出现的持续时间、状态转换的触发条件会同时受到内部、外部多种混杂因素共同影响。其影响机理复杂,可能超出领域知识的可解释范畴。
因此,如果想要结合参数化深度模型对具有跳变特性的多阶段系统以端对端的方式进行数据驱动建模,需要从探索\textbf{跳变系统阶段滞留时间及转移机制的可学习性}这一科学问题出发,
研究符合跳变系统先验特性的且具有阶段自识别、自转移能力的多阶段深度辨识模型。


具体地,该研究内容针对周期多阶段系统在不同阶段下动态特性彼此迥异,难以统一建模的问题,
研究了传递式多ODE-Net集成结构,以独立建模系统在不同阶段下的动态特性,并支持开环预测阶段转换处隐空间的状态衔接。
针对阶段持续时间难以预测、转换条件和位置难以识别的问题,
提出了跳变系统辨识问题中的自跳跃(Autonomous jump)概念,
并在多ODE-Net集成架构的基础上,研究了基于阶段持续时间预测网络的阶段自转移方法。
除此之外,针对工业系统的多输出项可能同时存在稳定特性和非稳定特性的情况,研究了不同类别输出项的解藕建模方法,并提出了稳定ODE与非稳定ODE相结合的分层常微分方程网络单元。最后,通过某真实工业制冷系统的的运行数据验证了本文提出方法在解决多输出周期跳变系统建模问题的有效性。


\subsection{基于深度常微分方程-马尔可夫模型的随机非确定性系统建模}
% 3) 基于常微分方程-循环状态空间模型的随机系统建模 \\

第三项研究内容“基于深度常微分方程-马尔可夫模型的随机非确定性系统建模”重点讨论了连续时间视角下,随机非确定性复杂系统建模的诸多问题。
现实世界中的复杂系统往往具备典型的随机性(Uncertainy)和非确定性(stochasticity),确定性系统辨识模型只能以最小化期望误差的方式拟合系统随机演化函数在某分布下的期望,不仅不便于实现系统的随机采样预测,且无法对可观测数据呈现出的非确定性和随机性给出表示与度量。因此,想要在辨识模型中引入对于系统随机非确定性的感知与描述能力,并给出概率域下的识别及预测结果,需要从“\textbf{部分可观测马尔可夫决策系统的随机性、非确定性产生机理}”这一科学问题出发,研究贝叶斯视角下隐空间状态的时序生成过程及逆向推理方法。

具体地,针对随机非确定性连续时间系统难以表征、建模的问题,
研究向确定性微分方程系统演化中添加随机转移路径的有效途径,
进而提出深度常微分方程与马尔可夫模型相结合的随机非确定性系统建模方法,并结合时序变分推断理论给出训练模型所需要优化的证据下界。
同时,针对训练阶段下原始时序变分推理算法不便于实现(Backpropagation through time, bptt)难以保证多步预测精度的问题,
研究了基于采样状态重用的高效隐空间超调技术,进而在不增加训练时间开销的情况下显著提升模型的开环预测精度。
最后,针对训练阶段批数据中不同常微分方程积分区间不一致导致难以并行化训练的问题,研究了基于重参数变换的批常微分方程并行求解技术,成功实现了不同积分区间下的高效并行训练。
最后,通过三个输入输出系统数据集验证了本文提出方法在解决随机非确定性系统建模及多步开环预测问题中的有效性。
\subsection{基于连续时间深度强化学习的复杂工业系统优化与控制}
第四项研究内容“基于连续时间深度强化学习的复杂工业系统优化与控制”重点讨论了连续时间视角下,基于数据驱动的复杂非线性工业系统的优化控制方法。
大部分复杂工业系统的运行过程具有不完全观测、非线性、多变量、高时滞等特点,
想要建立准确的数学模型描述其运转机理是极其困难的,因此基于模型的传统最优控制理论及方法难以适用。系统运行过程产生的历史数据为无模型数据驱动优化控制提供了可行的思路。
想要充分利用离线数据构建近似的系统模型并衍生形成可靠的控制策略,
需要从“数据驱动建模对于免模型强化学习的可改善性”这一科学问题出发,研究基于辨识模型指导的复杂工业系统强化学习控制方法。

具体地,针对非线性、高时滞复杂工业系统控制优化难的问题,研究了基于数据驱动的有模型自适应动态规划控制方法,定义了系统的状态空间、动作空间、效用函数、状态转移函数等内容,并提出了离线系统建模与在线强化学习相结合的原始控制器构建及生产环境在线学习策略,通过利用离线采集数据以及在线监测数据有效解决了复杂工业系统控制优化难的问题。
同时针对在线环境下策略网络增量训练开销大的,在线学习与控制难以满足实时性的问题,研究了基于自适应评价值迭代的控制动作求解算法,通过在模型架构中去掉策略网络有效减少了在线学习的计算开销。
另外,针对评价模块参数收敛慢、难以准确给出策略优化方向的问题,研究了基于短期经验回放的评价网络训练技术,有效提高了模型对于局部评价值梯度变化的敏感性及预测准确性。
最后,本文选用一种典型的复杂工业系统——尾矿浓缩机,利用其仿真模型验证了本文提出的自适应控制算法在控制精度、时间消耗方面的优势。
同时将该控制算法应用于某真实矿山的深锥浓密机底流浓度控制场景中,相比于原始的规则控制算法,大幅度提高了出料浓度的追踪控制效果及稳定性。

\section{论文的章节安排}

本文针对连续时间域下的复杂动态系统建模及控制中的关键技术开展研究,全文共分为六章,
第一章介绍本文工作的研究背景及意义,提出了基于数据驱动的复杂系统建模方法所面临的难点与挑战,并归纳了本文的主要创新点和贡献。
第二章到第五章是本文的主体部分,详细介绍了本文的所有研究内容。

% 第二章对于复杂系统的预测方法及深度微分方程网络的原理及技术进行介绍。

第二章首先介绍了复杂输入输出系统预测问题的形式化表述并给出基于状态空间模型的表示方法。
进一步地,介绍了本文提出的基于ODE-Net的连续时间输入输出系统预测模型,并分别介绍其序列编码器、导数模块以及状态解码器三大组成模块,同时给出了基于伴随状态的模型训练方法。
另外,该章针对短期预测和长期预测两种预测场景,给出了两种导数模块定义方法。
并提供了用于连续化系统输入序列的并行插值方法。
最后,该章介绍了将上述模型应用于膏体浓密机系统预测问题的实验结果。并从微分方程求解器的选择、序列编码器输入的长度设定等多个角度进行了消融实验。

第三章研究了基于有限状态机-常微分方程网络的周期性多阶段复杂系统预测建模技术,分别介绍了周期性多阶段系统的形式化定义,H-ODEnet结构设计,DFA-ODEnets模型结构设计、基于DFA-ODEnets的编码器解码器结构设计。

第四章研究了

第五章研究了

第六章对本文工作主要成果进行了总结,
最后面向对基于数据驱动的动态模型构建乃至世界模型构建问题,给出了几点关于模型设计与学习的研究展望。
