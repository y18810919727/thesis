% !Mode:: "TeX:UTF-8"

\chapter{引言}

\pagenumbering{arabic}
\setcounter{page}{1}

\section{研究背景与意义}
    
% 当前,新一轮科技革命和产业变革加速发展,新一代信息技术正在与工业生产 深度融合,自动化、信息化、智能化已经成为全球工业制造发展的重要方向。
% 2020年李克强总理发布的《政府工作报告》中,明确提出“推动制造业升级”、“推进智能制造”的要求。如何工业生产利用智能化技术,将,

% 为了实现工业4.0、工业互联网、数字孪生等伟大技术愿景,其中

复杂系统及设备的分析与优化技术涉及了集工业制造、自动控制、计算机技术等多学科知识,长久以来受到了国内外学者的广泛关注与深度研究。
如何建模复杂系统的运行过程并认知系统的内在机理,是实现系统分析优化的基础。
解决复杂系统的建模、预测问题旨在于对系统关键变量的自相关趋势以及受其他协变量的影响进行建模。
由于在工业环境下,复杂系统的运行过程存在非线性、高扰动、强耦合等复杂特性,从物理、化学、数学角度建模系统动态过程的传统分析方法不再适用。伴随着工业监测技术的不断完善以及生产自动化、信息化水平的不断提升,各种大型设备上安装了用于实时监测生产数据的传感器。数据量的与日剧增使得数据驱动方法成为建模复杂工业系统过程的主流方案。
基于数据驱动的复杂工业系统预测建模方法可以分成两大类:离散时间(Discrete Time, DT)系统建模和连续时间系统建模。
依托于近年来机器学习、强化学习\cite{sutton2018reinforcement}、深度学习\cite{lecun2015deep}\cite{duan2016}、时间序列分析技术\cite{shumway2000time}的发展与普及,从离散时间角度的建模研究方向涌现出了诸多成果,领域发展相对较为成熟,其原因与数字计算机将信息离散化的思想密不可分。相比而言,连续时间系统的相关理论体系研究较为滞后,且在当今大数据、数据驱动的时代背景下,并未与主流技术形成深度融合。但是对比离散时间系统建模方法,从连续时间域角度进行系统建模的思想在某些情况下具有较大优势:
\begin{itemize}
\item	对于物理属性的兼容性:大部分的物理现象都服从连续时间设定,因此利用连续时间模型对物理系统进行建模可以更好地匹配客观世界物理规律,且增加模型的可解释性。
\item	对于先验知识的表示:系统动态不同阶的相关度,如:位移、速度、加速度之间的相关性可以在连续时间系统下非常容易地表示出来,进而有效利用先验知识,降低模型求解的自由度。
\item	数据滤波:连续时间系统建模在处理有噪数据时具有天然的优势,而对于离散时间系统建模方法来说,去噪过程和建模过程完全分开独立的。
\item	非均匀的数据采样:当数据采样间隔不均匀时,离散时间方法无法使用,但是连续时间方法不受限于此。
\item	CT系统与DT系统的相互转换:对于CT系统可以通过变更采样率的方法获得任意时间间隔的DT系统,而DT系统则不然。当数据采样周期与下游应用所需的时间间隔不一致时,DT系统建模方法难以适用
\item	DT模型在高采样率下存在精确性问题:当数据采样率较高时,系统状态变化过程接近环形,不适宜的DT模型很容易存在精度差、鲁棒性差的问题;
\item	钢性系统:当系统同时存在快过程和慢过程时,离散采样的间隔选取不当很容易造成快过程或慢过程其中一类统计性质的丢失,对离散时间系统处理造成较大阻碍。
\end{itemize}

现有的基于机器学习的系统预测方法多从离散时间域角度描述系统动态过程,并利用数据驱动的方式对离散时间系统参数进行拟合训练。
但在复杂工业系统中,上述陈列的系统特性以及问题需求是时常存在的。
比如由于不同传感器工作频率不一致会导致数据存在非均匀采样情况,使用离散时间域模型之前需要对数据进行大量的前处理,这会对数据的原始特性造成损坏。

连续时间域模型对于复杂动态系统具有天然契合性,深度学习方法在参数化建模与复杂系统表示方面具有较大优势。
因此,本论文从二者结合的角度,对基于深度微分方程网络的复杂动态系统预测技术开展研究。
利用参数化的深度神经网络模型拟合复杂系统的微分方程,并基于拟合模型实现系统的预测、控制与优化。
使用连续时间域的深度神经网络模型拟合复杂系统,会面临以下研究难点与挑战:
1) 传统的连续时间系统辨识方法主要利用线性模型拟合系统的高阶微分方程。然而,这种方法不适用于建模具有长时延、高噪音、强非线性的不完全观测系统。
当使用深度神经网络拟合系统的连续时间动态时,如何有针对性地对网络进行设计以适用于上述复杂系统,是一个尚待解决的技术难题。另一方面,复杂微分方程的求解依赖于数值近似方法,数值近似求解过程中的大量计算使训练深度神经网络所需的计算图过于庞大。如何在保证数值求解精度的同时降低显存占用,也是将该理论方法应用于工程实践时亟待解决的关键问题。

2) 部分复杂工业系统在运行时呈现周期性多阶段特点,阶段的持续时间受多变量影响且不同阶段内呈现的系统动态特性具有极大差异。使用单一模型难以准确地拟合系统的所有阶段。另一方面,多输出系统中不同输出变量的时序特性存在较大差异,如何通过调整网络模型结构以对不同变量进行差异化处理也是亟待解决的研究问题。

针对上述问题与挑战,本文以具有连续时间动态特性的复杂系统作为研究对象,针对系统存在的非线性、非确定性、多阶段混合、高时延、不同输出量统计特征不一致等特性,将连续时间域模型的灵活性与深度神经网络的强大表示能力相结合,构建连续时间域下的深度模型以对系统进行预测。
本文所述工作为解决连续工业系统的预测、仿真、建模、优化问题提供了新的解决方案。
\section{研究内容与创新点}
具体的,本文研究内容及贡献点如下:
针对复杂系统存在的非线性、非确定性、多阶段混合、高时延、不同输出量统计特征不一致等特性,本文开展了如下几方面的研究:

1) 基于可微ODE-Net的高时延工业多输入输出系统预测 \\
针对复杂工业系统本质上为连续时间演化过程,且存在非线性、长时延等特性,本文提出以ODE-net作为模型骨架结构,从连续时间域角度拟合复杂工业系统的动态过程。采用一种由序列编码器、状态解码器和导数模块组成的深度连续时间(Continuous Time, CT )网络,以端到端的方式学习工业系统输出的自回归变化过程和输入对输出的非线性影响。以实现给定历史系统运行轨迹及未来系统控制输入序列情况下,预测系统输出的未来变化。

2) 基于有限状态机-常微分方程网络的周期性多阶段复杂系统建模 \\
为了建模具有多阶段周期转移特性的复杂系统,以及解决不同输出量统计特征不一致的问题。本文提出了一种基于有限状态机-常微分方程网络(DFA-ODEnets)的Encoder-Decoder 框架。
作为框架中的关键结构,DFA-ODEnets模型通过引入多个层次常微分方程网络(Hierarchical ODEnets,H-ODEnets),能够以数据驱动的方式独立地学习不同阶段下系统的动态演化特性。
同时模型中引入了阶段持续时间估计器,使其具备了用于长时间开环预测的阶段自转移能力。
该框架成功实现了某数据中心制冷系统的预测与仿真,并用于优化系统运行参数以降低制冷能耗。

3) 基于常微分方程-循环状态空间模型的非均匀采样随机系统建模 \\
对于同时具有非线性、非确定性(Uncertainy)及随机性(stochasticity)的复杂输入输出系统,
深度状态空间模型(SSMs)作为一种支持在线处理的参数模型,在学习状态推理和状态生成方面十分有效。
广泛应用于复杂系统的表示、预测和控制等问题中。
然而,大多数深度状态空间模型假设训练数据及预测数据是均匀采样的,当观测数据的采样时间非均匀时,离散时间状态空间模型无法适用。
为了解决这一问题,本章提出了一种新颖的连续时间深度状态空间模型——常微分方程循环状态空间模型(Ordinary Differential Equation Recurrent State Space Model, ODE-RSSM),该模型在状态转换过程中同时引入了随机和确定性路径,并利用ODE-Net建模连续时域下相邻采样时间点之间的隐状态变化。
另外,在训练阶段,为了使该模型能够支持批量并行推理与训练,需要解决batch中不同常微分方程积分区间不一致的问题。因此本章提出了一种批量常微分方程并行求解方法,能够高效地求解多个积分区间不同的常微分方程。
最后为了提升模型的长期预测精度,本章在模型训练阶段引入了采样状态重用的隐空间超调技术,相比于传统的隐状态超调技术能够以更低的训练开销提升长期预测精度。
最后,本章通过三个输入输出系统数据集对所述模型及改进进行评估。实验表明在非均匀采样下的随机系统预测任务上,ODE-RSSM无论是在处理非均匀采样方面还是建模系统随机性方面均具有良好表现,预测精度显著优于离散时间状态空间模型以及带有确定性状态演变的连续时间模型。
\section{论文的组织架构}

本文针对连续时间深度时序网络在复杂动态系统预测中的关键技术及应用开展研究,正文内容组织如下:
第一章介绍本文工作的研究背景及意义、研究内容与创新点、论文的组织结构。
第二章对于复杂系统的预测方法及深度微分方程网络的原理及技术进行介绍。
第三章研究了基于可微ODE-Net模型的高时延工业多输入输出系统预测技术。分别从问题形式化定义、基于ODE-net模型的预测框架设计、工业数据集验证三方面对该研究内容进行阐述。
第四章研究了基于有限状态机-常微分方程网络的周期性多阶段复杂系统预测建模技术,分别介绍了周期性多阶段系统的形式化定义,H-ODEnet结构设计,DFA-ODEnets模型结构设计、基于DFA-ODEnets的编码器解码器结构设计。
第五章对本文工作主要成果进行了总结,对该领域研究的下一步工作计划进行了展望。
