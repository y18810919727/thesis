% !Mode:: "TeX:UTF-8"

%中文摘要
\chapter*{ 摘 \space 要 }
\addcontentsline{toc}{chapter}{摘要}

采矿、化工、能源等过程工业领域中存在诸多具有非线性、高随机、长时滞、不完全观测特性的复杂动态系统。
依赖于机理分析的传统动态系统辨识方法难以充分拟合系统的高复杂性。
依靠大量训练数据驱动的参数化建模方法难以引入系统机理和先验知识,使得训练模型所需的数据量大,且模型在不同场景下泛化能力差。
系统的高复杂性与现有建模方法的局限性造成了\textbf{系统机理难分析、动态方程难辨识、关键指标难控制}等问题。
本文围绕连续时间域下复杂动态系统建模与优化这一问题展开讨论,重点研究了连续时间系统与深度学习的结合体——\text{深度微分方程网络}在复杂系统辨识与控制优化中的应用,所提出的模型及算法有效应用于膏体充填场景下的设备数据分析与管控过程中。
具体地,本文的主要研究工作包括:

1)针对具有非线性、长时延特性且本质上为连续时间过程的复杂工业系统,本文提出一种以ODE-Net作为骨架结构的深度连续时间系统辨识模型,模型能够从连续时间域角度学习工业系统输出的自回归变化和输入对输出的长时滞、非线性影响。经过膏体浓密机系统运行数据的验证,该模型在长时预测和短时预测场景下均获得较好效果。   

2)针对带有随机性的动态系统在非均匀采样设定下的系统辨识问题,
本文提出了常微分方程循环状态空间模型。该模型在基础循环神经网络的状态转移中引入随机路径以建模系统状态转移的随机性。同时,利用常微分方程网络建模相邻采样时间点之间的隐变量连续时间演化以支持非均匀采样设定下的模型训练。
另外,本文提出了一种求解批常微分方程的再参数化方法以解决时间间隔不均匀时微分方程难以并行求解、模型难以并行训练的问题。
利用膏体浓密机系统数据集和两个公有系统辨识数据集对模型进行验证,结果表明该模型在数据集存在随机性以及稀疏非均匀采样间隔时,能获得相比基线模型更好的建模效果。

3)针对复杂工业系统动态机理未知、难以控制优化问题,本文提出了基于连续时间有模型强化学习的启发式评价网络值迭代算法。该算法采用常微分方程网络构建动态系统的预测模型,并基于积分强化学习构建控制策略的自适应评价函数。
同时,依托于评价模块与模型网络的评价、预测能力,采用随机梯度下降法生成控制指令,使得整套控制模型支持高效的在线控制与参数更新。经过尾矿浓缩机仿真模型验证,该方法在非均匀数据采样下能够获得比其他控制算法更优的控制性能。同时,该算法成功部署应用于真实矿场的膏体浓密机控制系统中,相比原始的基于固定规则的控制策略获得了更好的浓度控制效果。

4)针对具有周期多阶段特性的动态系统辨识问题,本文提出了连续时间自跳变常微分方程网络。
在给定序列输入输出数据的所属阶段标注下,
该模型能够从非均匀采样数据中独立学习不同阶段下的系统动态特性。同时,模型引入阶段转换预测器以学习连续时间跳变系统的阶段转移过程,进而在模型开环预测时实现内部阶段变量的自转移。实验环节利用膏体制备过程中的水泥添加系统数据对模型进行验证,
结果表明模型能够在给定系统受控输入下,准确地预测阶段间的转移时间点以及各阶段内的系统输出,基于仿真结果成功优化膏体制备过程的水泥成本及设备开关机损耗成本8.2\%。


\vskip 30bp
{

    % 在关键词冒号后添加你的关键词,使用全角逗号分割
    \textbf{ \heiti \zihao{-4} 关键词:动态系统建模 ,系统辨识,常微分方程网络,有模型强化学习,浓密机,膏体充填}
}


%英文摘要

\chapter*{ Abstract }
\addcontentsline{toc}{chapter}{Abstract}
In the processing industry, most complex devices or systems have strong non-linearity, high stochasticity, and long time delay.
These properties lead to the difficulties in analyzing system mechanisms, identifying dynamic equations, and controling critical operational indices. 
Mechanistic-based dynamic system identification methods are not ideal to sufficiently fit the complexity of the system.
Deep learning methods ignore the limitations from system mechanisms and prior knowledge, 
which demands a larger size of training data and lacks generalization in different scenarios.
From the continuous-time domain, this dissertation focuses on the application of \text{deep differential equation networks}, a combination of continuous time systems and deep learning, in the identification and control optimization of complex systems.

(1) For the modeling of non-linear and long-delayed complex industrial systems which are essentially continuous-time processes, this dissertation proposes a continuous-time deep system identification model based on the ODE-net backbone.
The model learns the auto-regressive changes of industrial system outputs and the non-linear and time-delayed effects from inputs on outputs in continuous-time domain. 
The evaluation conducted on real industrial datasets of paste thickeners indicates that the model achieves comparable performances on both long-time and short-time predictions.   


(2) For the modeling of the complex systems with stochasticity, this dissertation proposes the  Ordinary Differential Equation Recurrent State Space Model(ODE-RSSM). 
On the basis of Recurrent Neural Network, an additional stochastic path defined by latent variable is embedded in the state transition for modeling stochasticity.
Furthermore, the ODE-RSSM incorporates an ordinary differential equation network to model the continuous-time evolution of latent states between adjacent time points.
Inspired from the equivalent linear transformation on integration limits,
this dissertation also propose an efficient reparameterization method for solving batched ODEs with non-uniform time spans in parallel for efficiently training the ODE-RSSM with irregularly sampled sequences.
Extensive experiments on three datasets, including the paste thickening dataset and two public system identification datasets, demonstrate that the ODE-RSSM outperforms other baselines in predicting stochastic system in open loop when the time spans of predicted points are uneven.


(3) For the controling of complex industrial systems without prior knowledge of mechanism, this dissertation proposes a novel
continuous-time model-based reinforcement learning control algorithm, Heuristic Critic Network Value Iteration (HCNVI).
The method models the dynamic system with differential equation neural network and introduces the integral reinforcement learning to approximate the adaptive critic function under the control law.
Depending on the evaluation of critic network and the prediction of model network, the controlling action is determined by stochastic gradient descent algorithms.
Experiments conducted on thickening simulation system verify that the proposed HCNVI outperforms the other control algorithms.
In the meanwhile, the method is also deployed in a thickening control system of a real paste backfilling station.
In comparison to the original rules-based controlling strategy, the algorithm is effective to control the underflow concentration in a stable range.


(4) To address the problem of modeling periodic multi-staged system, this dissertation proposes continuous-time Autonomous Jump Ordinary Differential Equation (AJ-ODE-Net) to learn continuous-time period jump systems.
Given sequential inputs and outputs belonged to labeled stages,
the model is able to independently learn the sub-dynamics in different stages from non-uniform sampled data. 
The model also introduces a stage transition predictor to realize the autonomous stage transformation in open-loop prediction. 
By evaluating the model with dataset from an cement  addition system in paste production, the results indicates that the AJ-ODE-Net can accurately predict the system outputs and the duration time in each stage.
The open-loop prediction of paste system runtime also assists in optimizing 8.2\% cost of backfilling.


\vskip 30bp
{
    \zihao{-4}
    % 在key words冒号后添加你的关键词,使用半角逗号加空格分割
    $\mathbf{Key}$ $\mathbf{Words}$: 
    $\mathbf{Dynamic\ system\ modeling}$, 
    $\mathbf{Model}$-$\mathbf{based\ reinforcement}$
    $\mathbf{learning}$,
    $\mathbf{System\ identification}$, 
    $\mathbf{Ordinary\ Differential\ Equation\ Neural\ Network}$,
    $\mathbf{Thickener}$,
    $\mathbf{Paste\ backfilling}$
    
}

    % \textbf{ \heiti \zihao{-4} 关键词:动态系统建模 ,系统辨识,常微分方程网络,有模型强化学习}
